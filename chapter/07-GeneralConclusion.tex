\chapter*{Conclusion et perspective}
\markboth{\MakeUppercase{Conclusion générale}}{}
\addcontentsline{toc}{chapter}{Conclusion générale}
\adjustmtc
\thispagestyle{MyStyle}

\justifying

\sloppy \setstretch{1.2} 

Ce rapport met en lumière l'importance cruciale de l'analyse des données dans le secteur des télécommunications, en particulier pour l'entreprise Ooredoo. Grâce à une méthodologie rigoureuse et à l'application de techniques avancées de data science, on a exploré les comportements des clients et identifié les facteurs clés influençant leur satisfaction.

Les résultats obtenus révèlent des corrélations significatives entre différentes variables, comme la consommation de minutes et l'accès aux données, ainsi que des tendances marquantes concernant l'ancienneté des clients et leur fréquence de recharge. Ces insights fournissent une base solide pour le développement de modèles prédictifs, permettant à Ooredoo d'anticiper les besoins des clients et d'adapter ses stratégies marketing de manière plus efficace.

Néanmoins, plusieurs problématiques ont émergé au cours de l'étude, ouvrant des perspectives d'amélioration pour les analyses futures. D'abord, la fiabilité des données collectées soulève des limites. L'envoi de messages comportant des questions spécifiques n'est pas toujours compris par l'ensemble des destinataires, dont les profils varient considérablement. Ces différences de compréhension peuvent altérer la qualité des réponses et, par conséquent, influencer la qualité des données. Il serait donc judicieux d'envisager une méthode de collecte plus standardisée et inclusive, par exemple en transformant les questions en un système de score de satisfaction plus accessible à tous.

Par ailleurs, certaines variables démographiques, telles que l'âge ou le genre, peuvent manquer de fiabilité, compromettant ainsi la précision des analyses basées sur ces critères. Cette incertitude peut introduire des biais dans les modèles prédictifs. Par conséquent, il est essentiel de renforcer la qualité des données démographiques afin d'assurer une segmentation plus précise et d'obtenir des résultats analytiques plus fiables.

En conclusion, cette étude ouvre la voie à de futures recherches qui devront perfectionner les méthodes de collecte des données et réduire les biais dans les modèles analytiques. Cela permettra aux entreprises de rester à la fois réactives et proactives face aux évolutions des comportements des consommateurs.
