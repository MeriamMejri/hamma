Table ~\ref{table:ISO Alpha} summarizes
%%%%%%%%%%%%%%%%%%%%%%%%%%%%%%%%%%%%%%%%%%%%%%%%%%%%%
%Define the Table Example 
%%%%%%%%%%%%%%%%%%%%%%%%%%%%%%%%%%%%%%%%%%%%%%%%%%%%%

\begin{table}[H]
\label{table:ISO Alpha}
\caption{Table Title}
\begin{tabular}{|p{3cm}|p{3cm}|p{3cm}|}
\hline
\rowcolor{lightgray} \multicolumn{3}{|c|}{Country List}     \\\hline
Country Name or Area Name & ISO ALPHA 2 Code & ISO ALPHA 3  \\\hline
Afghanistan               & AF               &AFG           \\\hline
\rowcolor{gray}
Aland Islands             & AX               & ALA          \\\hline
Albania                   &AL                & ALB          \\\hline
Algeria                   &DZ                & DZA          \\\hline
American Samoa            & AS               & ASM           \\\hline
Andorra                   & AD              & \cellcolor[HTML]{6195C9} AND    \\\hline
Angola                    & AO              & AGO            \\\hline

\end{tabular}
\label{table:ISO Alpha}
\end{table}




Lorem ipsum dolor sit amet, consectetur adipiscing elit. Sed non risus. Suspendisse lectus tortor, dignissim sit amet, adipiscing nec, ultricies sed, dolor\cite{titleBIB2}


\section{SECTION 1.4}
Test items => tiny bullet :
\renewcommand{\labelitemi}{\tiny$\bullet$}
\begin{itemize}[leftmargin=2cm, topsep=0pt]
        \begin{spacing}{1.25}
        \item item1
        \item item2
        \item item3
        \end{spacing}
\end{itemize}
\begin{spacing}{0.5}
\end{spacing}
Lorem ipsum dolor sit amet, consectetur adipiscing elit. Sed non risus. Suspendisse lectus tortor, dignissim sit amet, adipiscing nec, ultricies sed, dolor.

Test items => '-':
\renewcommand{\labelitemi}{\tiny$-$}
\begin{itemize}[leftmargin=2cm, topsep=0pt]
        \begin{spacing}{1.25}
        \item item1
        \item item2
        \item item3
        \end{spacing}
\end{itemize}


%%%%%%%%%%%%%%%%%%%%%%%%%%%%%%%%%%%%%%%%
%%%%%%%%%%%%%%%%%%%%%%%%%%%%%%%%%%%%%%%%
%%%%%%%%%%%%%%%%%%%%%%%%%%%%%%%%%%%%%%%%
%%%%%%%%%%%%%%%%%%%%%%%%%%%%%%%%%%%%%%%%


\begin{figure}[H]
    \centering
    \subfloat[\centering Log parsed with filebeat]{{\includegraphics[width=4cm]{images/filebeat.png} }}%
    \qquad
    \subfloat[\centering Normal logs]{{\includegraphics[width=7cm]{images/normal log.png} }}%
    \caption{Difference between logs }%
    \label{fig:example}%
\end{figure}
%%%%%%%%%%%%%%%%%%%%%%%%%%%%%%%%%%%%%%
%%%%%%%%%%%%%%%%%%%%%%%%%%%%%%%%%%%%%%
%%% table de fpga %%%%%%%%%%%%%%%%%%%%
\begin{landscape}
\begin{table}[H]
\centering
\resizebox{\columnwidth}{!}{%
\begin{tabular}{|l|l|l|l|l|}
\hline
Characteristic                                                                      & FPGA                                                                                                                          & SoC                                                                                                                                             & ASIC                                                                                                                                      & Design Verification                                                                                                                 \\ \hline
Configurability                                                                     & \begin{tabular}[c]{@{}l@{}}Highly configurable, \\ users can define custom \\ digital logic circuits.\end{tabular}            & \begin{tabular}[c]{@{}l@{}}Less configurable, \\ as components are \\ typically fixed and \\ tailored to \\ specific applications.\end{tabular} & \begin{tabular}[c]{@{}l@{}}Fixed, designed for\\ a single specific \\ purpose, not \\ reconfigurable.\end{tabular}                        & \begin{tabular}[c]{@{}l@{}}Used to validate designs \\ for various integrated circuits.\end{tabular}                                \\ \hline
\begin{tabular}[c]{@{}l@{}}Hardware \\ Description \\ Languages (HDLs)\end{tabular} & \begin{tabular}[c]{@{}l@{}}Programmed using HDLs \\ such as VHDL or Verilog.\end{tabular}                                     & \begin{tabular}[c]{@{}l@{}}May involve HDLs, but often \\ involves software development \\ for specific applications.\end{tabular}              & \begin{tabular}[c]{@{}l@{}}Custom-designed and may not \\ involve HDLs.\end{tabular}                                                      & \begin{tabular}[c]{@{}l@{}}Used to ensure that the hardware \\ design meets specifications and \\ functions correctly.\end{tabular} \\ \hline
Reprogrammable                                                                      & \begin{tabular}[c]{@{}l@{}}Reprogrammable multiple \\ times to adapt to changing \\ requirements.\end{tabular}                & \begin{tabular}[c]{@{}l@{}}Typically not reprogrammable \\ after manufacturing.\end{tabular}                                                    & \begin{tabular}[c]{@{}l@{}}Not reprogrammable, fixed \\ at the time of manufacturing.\end{tabular}                                        & \begin{tabular}[c]{@{}l@{}}Not applicable; it is a process \\ to validate hardware designs.\end{tabular}                            \\ \hline
Parallel Processing                                                                 & \begin{tabular}[c]{@{}l@{}}Excels in parallel processing, \\ suitable for applications \\ requiring parallelism.\end{tabular} & \begin{tabular}[c]{@{}l@{}}Capable of parallel processing\\ but often less flexible \\ compared to FPGAs.\end{tabular}                          & \begin{tabular}[c]{@{}l@{}}Specific to the application, \\ may or may not support \\ parallel processing.\end{tabular}                    & \begin{tabular}[c]{@{}l@{}}Focuses on validation rather \\ than processing capabilities.\end{tabular}                               \\ \hline
Rapid Prototyping                                                                   & \begin{tabular}[c]{@{}l@{}}Used for rapid prototyping \\ and experimentation with \\ different designs.\end{tabular}          & \begin{tabular}[c]{@{}l@{}}Less suitable for rapid \\ prototyping, as the \\ components are usually fixed.\end{tabular}                         & \begin{tabular}[c]{@{}l@{}}Not used for rapid prototyping \\ due to the fixed nature of the design.\end{tabular}                          & \begin{tabular}[c]{@{}l@{}}Ensures hardware designs are \\ correct before manufacturing.\end{tabular}                               \\ \hline
\begin{tabular}[c]{@{}l@{}}Digital Signal \\ Processing (DSP)\end{tabular}          & \begin{tabular}[c]{@{}l@{}}Well-suited for DSP \\ applications due to \\ parallel processing capabilities.\end{tabular}       & \begin{tabular}[c]{@{}l@{}}Can handle DSP but may \\ not be as efficient as FPGAs.\end{tabular}                                                 & \begin{tabular}[c]{@{}l@{}}Custom-designed ASICs can \\ be optimized for DSP tasks.\end{tabular}                                          & \begin{tabular}[c]{@{}l@{}}Part of the validation process \\ to ensure DSP components \\ work as intended.\end{tabular}             \\ \hline
Embedded Systems                                                                    & \begin{tabular}[c]{@{}l@{}}Often used in embedded \\ systems where configurability \\ is a benefit.\end{tabular}              & \begin{tabular}[c]{@{}l@{}}Commonly used in embedded \\ systems, combining various \\ components on a single chip.\end{tabular}                 & \begin{tabular}[c]{@{}l@{}}Custom ASICs can be used in \\ embedded systems for \\ specific functions.\end{tabular}                        & \begin{tabular}[c]{@{}l@{}}Ensures the hardware design \\ works correctly within an \\ embedded system.\end{tabular}                \\ \hline
Cryptography                                                                        & \begin{tabular}[c]{@{}l@{}}Used in cryptographic \\ applications for secure \\ and efficient processing.\end{tabular}         & \begin{tabular}[c]{@{}l@{}}Can be used for cryptography, \\ but may require additional \\ components.\end{tabular}                              & \begin{tabular}[c]{@{}l@{}}Custom ASICs can be \\ optimized for encryption/decryption.\end{tabular}                                       & \begin{tabular}[c]{@{}l@{}}Part of the validation process \\ to ensure cryptographic \\ components are accurate.\end{tabular}       \\ \hline
Cost Difference                                                                     & \begin{tabular}[c]{@{}l@{}}Generally moderate cost, \\ depending on the complexity \\ of the design.\end{tabular}             & \begin{tabular}[c]{@{}l@{}}Cost varies widely depending \\ on the components and integration.\end{tabular}                                      & \begin{tabular}[c]{@{}l@{}}Higher upfront design and \\ manufacturing costs but lower \\ per-unit costs for mass production.\end{tabular} & \begin{tabular}[c]{@{}l@{}}Part of the cost control process \\ for hardware development.\end{tabular}                               \\ \hline
\begin{tabular}[c]{@{}l@{}}High-Performance \\ Computing\end{tabular}               & \begin{tabular}[c]{@{}l@{}}Utilized in high-performance \\ computing for specific \\ parallel tasks.\end{tabular}             & \begin{tabular}[c]{@{}l@{}}May be used in HPC but \\ often lacks the flexibility \\ of FPGAs.\end{tabular}                                      & \begin{tabular}[c]{@{}l@{}}Custom ASICs can be designed \\ for high-performance computing.\end{tabular}                                   & \begin{tabular}[c]{@{}l@{}}Ensures the hardware design \\ is capable of high-performance \\ tasks.\end{tabular}                     \\ \hline
\end{tabular}%
}
\caption{Comparison of FPGA, SoC, ASIC, and Design Verification: Characteristics}
\label{tab:my-table}
\end{table}
\end{landscape}