\chapter*{Résumé}

\addcontentsline{toc}{chapter}{Résumé}
\adjustmtc
\thispagestyle{MyStyle}

\justifying

\sloppy \setstretch{1.2} 

Ce rapport présente une analyse approfondie des données clients dans le secteur des télécommunications, avec un accent particulier sur l'impact des variables quantitatives et qualitatives sur la consommation et la satisfaction des clients. Voici les points clés abordés :

\begin{itemize}
    \item \textbf{Contexte et Objectifs} : Ce projet vise à analyser les comportements des clients d'une entreprise de télécommunications en utilisant des techniques avancées de data science. L'objectif est de mieux comprendre les besoins des clients pour améliorer les services et les offres proposées.
    
    \item \textbf{Méthodologie} : L'analyse repose sur des outils statistiques, notamment les tests de corrélation tels que l'ANOVA et Kruskal-Wallis, pour examiner les liens entre diverses variables quantitatives et qualitatives. L'étude inclut l'analyse de la consommation de données, l'usage du type d'appareil et l'accès à la 4G.
    
    \item \textbf{Résultats} : Des corrélations significatives ont été mises en évidence, notamment une relation positive entre la consommation de minutes et l'accès à la 4G. En revanche, l'ancienneté des clients est négativement corrélée avec la fréquence de recharge, indiquant une baisse d'activité des clients plus anciens.
    
    \item \textbf{Conclusion} : L'analyse des données révèle des tendances importantes dans les comportements clients, fournissant des informations précieuses pour l’orientation des stratégies marketing, et pour l'amélioration de la satisfaction client.
    
    \item \textbf{Implications} : Les résultats de cette étude offrent des pistes pour l’optimisation des offres de services, la personnalisation des communications client, et, à terme, une meilleure fidélisation et satisfaction des utilisateurs.
\end{itemize}

\chapter*{Abstract}
\addcontentsline{toc}{chapter}{Abstract}
This report presents an in-depth analysis of customer data in the telecommunications sector, with a particular focus on the impact of quantitative and qualitative variables on customer consumption and satisfaction. The key points covered are:

\begin{itemize}
    \item \textbf{Context and Objectives} : This project aims to analyze the behavior of telecommunications customers using advanced data science techniques. The goal is to better understand customer needs in order to improve services and offers.
    
    \item \textbf{Methodology} : The analysis is based on statistical tools, including correlation tests such as ANOVA and Kruskal-Wallis, to examine the relationships between various quantitative and qualitative variables. The study includes the analysis of data consumption, device usage, and 4G access.
    
    \item \textbf{Results} : Significant correlations were identified, including a positive relationship between minutes consumption and 4G access. In contrast, customer tenure is negatively correlated with recharge frequency, indicating reduced activity among older customers.
    
    \item \textbf{Conclusion} : Data analysis reveals important trends in customer behavior, providing valuable insights for guiding marketing strategies and improving customer satisfaction.
    
    \item \textbf{Implications} : The results of this study offer avenues for optimizing service offerings, personalizing customer communications, and ultimately improving user loyalty and satisfaction.
\end{itemize}

