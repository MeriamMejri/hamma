\chapter*{INTRODUCTION \MakeUppercase{générale}}
\markboth{\MakeUppercase{INTRODUCTION générale}}{}
%\addstarredchapter{GENERAL INTRODUCTION}
\addcontentsline{toc}{chapter}{INTRODUCTION \MakeUppercase{générale}}
\adjustmtc
\thispagestyle{MyStyle}



% In the era of digitalization, understanding the details of a particular company becomes necessary to guarantee reliability and shore up base. This internship aims towards company level assessment by collecting the essential facts concerning the legal presence, finances, brand, and activities of some organizations. We will look at these assessments including both automated and manual attempts and stress the automated ones.

% Additionally, the project involves developing a Bubble.io app to allow users to create and maintain their own forms as well as use standard ones from the library. This project involves user access rights management, user pages creation, templates organization, and access control configuration. It is anticipated that this procedure will improve users ‘and agencies’ form management work.
% \section*{Context and Motivation}
% In today's digital age, getting a full picture of companies is crucial to ensure their trustworthiness and protect operations. Businesses now depend more and more on precise and current info about third parties, including their legal status financial well-being, brand image, and day-to-day activities. The capacity to evaluate companies and plays a key role in keeping business secure and meeting regulatory requirements. This internship project aims to improve third-party risk management by focusing on company-level assessments. These assessments gather key information through both manual and automated processes, with a special focus on automation to boost productivity and cut down on human mistakes. This two-pronged approach allows for stronger evaluations and speeds up decision-making, which in turn boosts an organization's ability to manage risks. But old-school ways of collecting and evaluating data manual efforts often take a lot of time and resources. As a result more people want automated tools that can help collect and assess data. This has created a need for an easy-to-use platform that helps users create questionnaires, evaluate answers, and handle assessment workflows.

% \section*{The Need for the Platform}
% Collecting and evaluating company data is complex, which means we need a platform to help gather information and automate much of the evaluation process. The platform created during this internship tackles these challenges by helping users create standardized questionnaires to assess companies. By building automation into the platform, we can make evaluations more accurate. This cuts down on subjective interpretations and makes sure we capture all the necessary data points in the same way every time. The platform helps create questionnaires to check important things like legal status, money matters, day-to-day operations, and how people see the brand. These automatic tools show users how to get the info they need while also looking at the answers using AI smarts. In the end, this makes the whole process work better making it faster and more accurate to figure out risks from other companies.


% \section*{Project Objective and Contribution}
% The main goal of this project is to build a user-friendly app with \textbf{\textbf{Bubble.io}}. This app will let users make, handle, and keep up their own forms. It will also give them access to ready-made templates from a shared library. The platform aims to make form creation and management easier for users. It focuses on user access rights, page creation, and \textbf{\textbf{templates organization}}.
% This project also brings in several \textbf{\textbf{AI-enhanced tools}}. These tools help users evaluate the answers to these questionnaires. The platform automates key parts of the risk evaluation process. This gives users valuable insights that would need a lot of manual analysis. As a result, users have less work to do. At the same time, assessments become more consistent and accurate.
% When it comes to what this platform brings to the table:
% \begin{itemize}
%     \item It has an impact on \textbf{\textbf{user access rights management}} making sure users can see and handle their own forms, which boosts security and privacy.
%     \item It provides \textbf{\textbf{user-specific pages}} to manage forms, which makes the whole experience better by giving each user their own interface.
%     \item It sets up templates in an organized way, so users can find and tweak pre-made forms.
% \end{itemize}
% By the time this internship wraps up, we think the platform will boost how well companies do assessments and make risk management smoother for organizations. This means it'll be a key tool to beef up security and help with following the rules.



\textbf{L'industrie des télécommunications} a connu une évolution rapide au cours des dernières décennies, devenant un pilier central de la société moderne. Ce secteur, autrefois centré sur la simple fourniture de services de téléphonie et de communication, s'est progressivement diversifié pour offrir une multitude de services numériques à une clientèle de plus en plus exigeante. Dans ce contexte, le marketing est devenu un élément clé pour les entreprises de télécommunications, leur permettant de comprendre et d'anticiper les besoins de leurs clients afin de rester compétitives.\par

Historiquement, le marketing dans le secteur des télécommunications reposait sur l'intuition, l'expérience, et l'analyse des tendances par des experts en marketing. Les stratégies étaient souvent basées sur des études de marché traditionnelles, des sondages et des analyses qualitatives. Cependant, avec l'explosion des données numériques et l'avènement des technologies de la data science, ce paradigme a radicalement changé.\par
Aujourd'hui, l'exploitation des données occupe une place au coeur des stratégies marketing du secteur des télécommunications. Les entreprises ne se contentent plus de comprendre les comportements des clients à travers des méthodes traditionnelles, mais utilisent des techniques avancées de data science pour extraire des insights à partir de vastes quantités de données. Ces technologies permettent non seulement de décrire les comportements passés, mais aussi de prédire les actions futures des clients, facilitant ainsi la prise de décision et l'optimisation des stratégies marketing.

L'intégration de la data science dans le marketing des télécommunications a ouvert de nouvelles perspectives, transformant la manière dont les entreprises interagissent avec leurs clients. \textbf{L'analyse prédictive}, en particulier, permet de modéliser et d'anticiper l\textbf{es scores de satisfaction} des clients, offrant ainsi aux entreprises un levier puissant pour améliorer leurs services, fidéliser leurs clients et développer des offres sur mesure.Cette révolution numérique continue de remodeler le secteur, rendant les stratégies marketing plus précises, personnalisées et plus efficaces.\par

Cette recherche s'insère dans une analyse détaillée ayant pour objectif de mieux comprendre le comportement des consommateurs \textbf{d'Ooredoo} et à développer \textbf{un modèle statistique} capable de prédire les réponses aux enquêtes de satisfaction. \par

En utilisant \textbf{des outils mathématiques} \textbf{et statistiques} avancés, ce projet cherche à fournir des solutions pour anticiper les perceptions et réactions des clients, permettant ainsi d'améliorer les services offerts par \textbf{Ooredoo }et de renforcer sa position sur le marché des télécommunications. 

 








