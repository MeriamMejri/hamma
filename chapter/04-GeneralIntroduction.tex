\chapter*{INTRODUCTION \MakeUppercase{générale}}
\markboth{\MakeUppercase{INTRODUCTION générale}}{}
%\addstarredchapter{GENERAL INTRODUCTION}
\addcontentsline{toc}{chapter}{INTRODUCTION \MakeUppercase{générale}}
\adjustmtc
\thispagestyle{MyStyle}


\justifying

\sloppy \setstretch{1.2} 

\textbf{L'industrie des télécommunications} a connu une évolution rapide au cours des dernières décennies, devenant un pilier central de la société moderne. Ce secteur, autrefois centré sur la simple fourniture de services de téléphonie et de communication, s'est progressivement diversifié pour offrir une multitude de services numériques à une clientèle de plus en plus exigeante. Dans ce contexte, le marketing est devenu un élément clé pour les entreprises de télécommunications, leur permettant de comprendre et d'anticiper les besoins de leurs clients afin de rester compétitives.\par

Historiquement, le marketing dans le secteur des télécommunications reposait sur l'intuition, l'expérience, et l'analyse des tendances par des experts en marketing. Les stratégies étaient souvent basées sur des études de marché traditionnelles, des sondages et des analyses qualitatives. Cependant, avec l'explosion des données numériques et l'avènement des technologies de la data science, ce paradigme a radicalement changé.\par

Aujourd'hui, l'exploitation des données occupe une place au cœur des stratégies marketing du secteur des télécommunications. Les entreprises ne se contentent plus de comprendre les comportements des clients à travers des méthodes traditionnelles, mais utilisent des techniques avancées de data science pour extraire des insights à partir de vastes quantités de données. Ces technologies permettent non seulement de décrire les comportements passés, mais aussi de prédire les actions futures des clients, facilitant ainsi la prise de décision et l'optimisation des stratégies marketing.

L'intégration de la data science dans le marketing des télécommunications a ouvert de nouvelles perspectives, transformant la manière dont les entreprises interagissent avec leurs clients. \textbf{L'analyse prédictive}, en particulier, permet de modéliser et d'anticiper les \textbf{scores de satisfaction} des clients, offrant ainsi aux entreprises un levier puissant pour améliorer leurs services, fidéliser leurs clients et développer des offres sur mesure. Cette révolution numérique continue de remodeler le secteur, rendant les stratégies marketing plus précises, personnalisées et plus efficaces.\par

Cette recherche s'insère dans une analyse détaillée ayant pour objectif de mieux comprendre le comportement des consommateurs \textbf{d'Ooredoo} et à développer \textbf{un modèle statistique} capable de prédire les réponses aux enquêtes de satisfaction.\par

En utilisant \textbf{des outils mathématiques} \textbf{et statistiques} avancés, ce projet cherche à fournir des solutions pour anticiper les perceptions et réactions des clients, permettant ainsi d'améliorer les services offerts par \textbf{Ooredoo} et de renforcer sa position sur le marché des télécommunications.\par

Dans le cadre de cette étude, la démarche suivie peut être résumée comme suit. Le rapport commence par une présentation générale du projet, qui introduit les objectifs, les enjeux, ainsi que le cadre d'étude en lien avec le comportement des clients dans le secteur des télécommunications. Ensuite, une exploration des outils mathématiques et des méthodes statistiques est présentée, avec un accent particulier sur les techniques d'analyse de corrélation et les tests statistiques. La troisième étape est dédiée à la réalisation du projet, où sont décrites les étapes de collecte et de traitement des données, ainsi que la mise en place du modèle prédictif visant à estimer les scores de satisfaction des clients. Enfin, le rapport se termine par la présentation des résultats obtenus à travers l'analyse, accompagnée de leur interprétation afin de dégager des conclusions et recommandations concrètes pour la stratégie marketing d'Ooredoo.

Cette démarche permet de structurer l'analyse en étapes clés, offrant ainsi une vue claire du processus suivi pour atteindre les objectifs de l'étude.








